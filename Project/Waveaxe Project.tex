\documentclass{article}
\usepackage{amsmath}
\usepackage{amssymb}
\usepackage{graphicx}
\usepackage{todonotes}
\usepackage{stmaryrd}
\allowdisplaybreaks
\usepackage[backend=biber,
bibencoding=utf8,
bibstyle=ieee,citestyle=numeric-comp,
dashed=false,
maxcitenames=2,
mincitenames=1,
date=year,
doi=false,
isbn=false,
url=false, 
]{biblatex}
\addbibresource{unique.bib}

\begin{document}
	\title{Hydraulic fracturing with pressure pulses}
	\maketitle
Project steps:
\begin{itemize}
	\item Governing equations
		\subitem Show that we don't need a pressure boundary condition$^1$ \checkmark
		\subitem Non-dimensionalize the equations$^1$
		\subitem Incorporate porous reservoir fluid$^4$
	\item Implementation
		\subitem Clean up the existing code$^1$
		\subitem Make sure the same equations are being solved with the code$^1$
		\subitem Implement porous media model$^4$
	\item Verify the model
		\subitem Run some quasi-static planar fracture growth models to verify with analytical solution$^1$
		\subitem Verify porous media flow with results from the literature$^4$
	\item Piston model
		\subitem Model pressure pulses as step loads of constant injection$^1$
		\subitem Compute injection rate of each pulse from piston model$^2$
	\item Final report simulations
		\subitem Select parameters$^1$
		\subitem Set up config file for the report$^1$$^2$$^4$
		\subitem Run simulation$^1$$^2$$^4$
		\subitem Modify model until we have results we're happy with$^2$$^4$
	\item Parameter study
		\subitem Read through Ali's paper to see methodology$^1$
		\subitem Decide on parameters to study$^1$
		\subitem Select parameter range$^1$
		\subitem Set up config files$^3$$^4$
		\subitem Run simulations$^3$$^4$
	\item Design charts$^3$
	\item Report writing
		\subitem Introduction$^1$
		\subitem Methodology$^1$
		\subitem Results: dynamic pressure pulsing, parameter study$^3$$^4$
		\subitem Design charts$^3$$^4$
		\subitem Conclusions$^4$
\end{itemize}

Project stages: 
\begin{enumerate}
	\item First pass at getting results
	\item Figure out piston model that we're happy with
	\item Parameter study and design charts
	\item Porous media model
\end{enumerate}

\section{Governing Equations}

\begin{equation}
\nabla \cdot \boldsymbol\sigma = \mathbf{0}
\end{equation}

\begin{equation}
\boldsymbol \sigma = \mathbb{C} \colon \boldsymbol \varepsilon 
\end{equation}

\begin{align}
\mathbf{u}(\mathbf{x},t)\cdot \mathbf{n} &= 0 &\forall \: \mathbf{x} \in \Gamma_u, \: t \geq 0
\\
\boldsymbol\sigma \cdot \mathbf{n} &= p_w(t)\mathbf{n} &\forall \: \mathbf{x} \: \in \Gamma_w, \: t \geq 0
\end{align}

\begin{equation}
w = \llbracket \mathbf{u}\rrbracket \cdot \mathbf{n}
\end{equation}

\begin{equation}
\dfrac{\partial w}{\partial t} + \dfrac{\partial q}{\partial s} + Q_{leak}(s, t) = 0
\end{equation}

\begin{equation}
q = \dfrac{-w^3}{12\mu}\dfrac{\partial p}{\partial s} 
= -k \dfrac{\partial p}{\partial s}
\end{equation}

\begin{align}
q(s_{inlet}) &= Q_{inj}(t)
\\
q(s_{tip}) &= 0
\end{align}

\subsection{Discrete equations}
Semi-discretized rock deformation	
\begin{equation}
\label{eqn:solid-discrete}
\begin{aligned}
\underbrace{\left[\int\limits_{\Omega_s} \mathbf{B}^{\top} \mathbf{D} \mathbf{B} d\Omega \right] \mathbf{d}^n}_{\mathbf{F}^{int}}
&+ 
\underbrace{
	\left[\int\limits_{\Gamma_{c}} \llbracket \mathbf{N} \rrbracket ^{\top} \left(t_c^n \mathbf{I} - \boldsymbol\sigma_0 \right) \hat{\mathbf{n}}_{\Gamma_c} d\Gamma \right]	
}_{\mathbf{F}^{c}}
- 
\underbrace{\left[\int\limits_{\Gamma_{c}} \llbracket \mathbf{N} \rrbracket ^{\top} \hat{\mathbf{n}}_{\Gamma_c} p^n d\Gamma \right] 
	 }_{\mathbf{F}^{p}}
\\
&- 
\underbrace{\left[\int\limits_{\Gamma_w} \mathbf{N}^{\top} \left( p_w^n \hat{\mathbf{n}}_{w} + \boldsymbol\sigma_0\hat{\mathbf{n}}_{w} \right) d\Gamma \right]}_{\mathbf{F}^{ext}}
= \mathbf{0}
\end{aligned}
\end{equation}

Semi-discretized fluid flow
	\begin{equation}
\label{eqn:fluid-discrete}
\begin{gathered}
\underbrace{\left[\int\limits_{\Omega_c}
	\mathbf{B}_f^\top q^n ds 
	\right]}_{\mathbf{F}^d}
-
\dfrac{1}{\Delta t^n}
\underbrace{\left[
	\int\limits_{\Omega_c} \mathbf{N}_f^\top \Delta w^n ds 
	\right]}_{\mathbf{F}^s}
-
\underbrace{\left[
	\int\limits_{\Omega_c} \mathbf{N}_f^{\top} Q_{leak}^n ds
	\right]}_{\mathbf{F}^l}
	\\
-
\underbrace{\left[
	\left(\mathbf{N}_f^{\top} Q_{inj} \right) \Big\rvert_{\Gamma_q} 
	\right]}_{\mathbf{F}^f}
= \mathbf{0} 
\end{gathered}
\end{equation}


Define pressure and aperture: 
\begin{align}
	w &= \hat{\mathbf{n}}_{\Gamma_c}^{\top} \llbracket\mathbf{N}\rrbracket \mathbf{d}
	\\
	p &= \mathbf{N}_f \mathbf{p}
\end{align}

The fluid flux is defined with the cubic law as
\begin{equation}
q = -k \mathbf{B}_f \mathbf{p}
\end{equation}

The wellbore pressure is defined as
\begin{equation}
	p_w = \bar{\mathbf{I}}_1^{\top} \mathbf{p}
\end{equation}
in which $\bar{\mathbf{I}}_1$ is a column matrix with the only non-zero component is the first component (1).

Discretized rock deformation	
\begin{equation}
\begin{aligned}
\underbrace{\left[\int\limits_{\Omega_s} \mathbf{B}^{\top} \mathbf{D} \mathbf{B} d\Omega \right] \mathbf{d}^n}_{\mathbf{F}^{int}}
&+ 
\underbrace{
	\left[\int\limits_{\Gamma_{c}} \llbracket \mathbf{N} \rrbracket ^{\top} \left(t_c^n \mathbf{I} - \boldsymbol\sigma_0 \right)\hat{\mathbf{n}}_{\Gamma_c} d\Gamma \right]	
}_{\mathbf{F}^{c}}
- 
\underbrace{\left[\int\limits_{\Gamma_{c}} \llbracket \mathbf{N} \rrbracket ^{\top} \hat{\mathbf{n}}_{\Gamma_c} \mathbf{N}_f d\Gamma \right] \mathbf{p}^n
}_{\mathbf{F}^{p}}
\\
&- 
\underbrace{\left[\int\limits_{\Gamma_w} \mathbf{N}^{\top} \hat{\mathbf{n}}_{w} d\Gamma \:
\bar{\mathbf{I}}^{\top}\mathbf{p}^n
+ 
\int\limits_{\Gamma_w} \mathbf{N}^{\top} \boldsymbol\sigma_0\hat{\mathbf{n}}_{w} d\Gamma
 \right]}_{\mathbf{F}^{ext}}
= \mathbf{0}
\end{aligned}
\end{equation}


Discretized fluid flow
\begin{equation}
\begin{gathered}
\underbrace{\left[\int\limits_{\Omega_c}
	\mathbf{B}_f^\top k^n \mathbf{B}_f ds \right] \mathbf{p}^n
	}_{\mathbf{F}^d}
+
\dfrac{1}{\Delta t^n}
\underbrace{\left[
	\int\limits_{\Omega_c} \mathbf{N}_f^\top (w^n - w^{n-1})  ds 
	\right]}_{\mathbf{F}^s}
+
\underbrace{\left[
	\int\limits_{\Omega_c} \mathbf{N}_f^{\top} Q_{leak}^n ds
	\right]}_{\mathbf{F}^l}
	\\
+
\underbrace{\left[
	\left(\mathbf{N}_f^{\top} Q_{inj}^n \right) \Big\rvert_{\Gamma_q} 
	\right]}_{\mathbf{F}^f}
= \mathbf{0} 
\end{gathered}
\end{equation}


Coupled system: 
	\begin{equation}
\begin{aligned}
\mathbf{R}_d^n &= \mathbf{F}^{int}(\mathbf{d}^n) 
+ \mathbf{F}^{c}(\mathbf{d}^n) 
- \mathbf{F}^{p}(\mathbf{p}^n) 
- \mathbf{F}^{ext}(\mathbf{p}^n) = \mathbf{0}\\
\mathbf{R}_p^n &= \mathbf{F}^{d}(\mathbf{d}^n, \mathbf{p}^n, \boldsymbol{\phi}^n) 
+ \dfrac{1}{\Delta t} \mathbf{F}^{s}(\Delta\mathbf{d}, \Delta \boldsymbol\phi)
+\mathbf{F}^{l} 
+ \mathbf{F}^{f} = \mathbf{0}
\end{aligned}
\label{eqn:system}
\end{equation}

\section{Matrix form}

Discretized rock deformation	
\begin{equation}
\begin{aligned}
\underbrace{\left[\int\limits_{\Omega_s} \mathbf{B}^{\top} \mathbf{D} \mathbf{B} d\Omega \right] \mathbf{d}^n}_{\mathbf{F}^{int} = \mathbf{K}_{dd} \mathbf{d}^n}
&+ 
\underbrace{
	\left[\int\limits_{\Gamma_{c}} \llbracket \mathbf{N} \rrbracket^{\top} \hat{\mathbf{n}}_{\Gamma_c} \dfrac{\partial t_c^n}{\partial w^n} \hat{\mathbf{n}}_{\Gamma_c}^{\top}\llbracket \mathbf{N} \rrbracket 
	  d\Gamma \right]	\mathbf{d}^n
}_{\mathbf{F}^{c} = \mathbf{K}_c \mathbf{d}^n}
+ 
\underbrace{
	\left[\int\limits_{\Gamma_{c}} \llbracket \mathbf{N} \rrbracket^{\top} 
	\left( t_{c_0}^n \mathbf{I} - \boldsymbol\sigma_0   \right) \hat{\mathbf{n}}_{\Gamma_c}
	d\Gamma \right]
}_{\mathbf{F}^{c0}}
\\
- 
\underbrace{\left[\int\limits_{\Gamma_{c}} \llbracket \mathbf{N} \rrbracket ^{\top} \hat{\mathbf{n}}_{\Gamma_c} \mathbf{N}_f d\Gamma \right] \mathbf{p}^n
}_{\mathbf{F}^{p} = \mathbf{K}_{dp} \mathbf{p}^n}
&- 
\underbrace{\left[\int\limits_{\Gamma_w} \mathbf{N}^{\top} \hat{\mathbf{n}}_{w} d\Gamma \:
	\bar{\mathbf{I}}^{\top} \right]\mathbf{p}^n
}_{\mathbf{F}^{ep} = \mathbf{K}_{ep} \mathbf{p}^n}
	-
\underbrace{\left[
	\int\limits_{\Gamma_w} \mathbf{N}^{\top} \boldsymbol\sigma_0\hat{\mathbf{n}}_{w} d\Gamma
\right]
}_{\mathbf{F}^{ext}}
= \mathbf{0}
\end{aligned}
\end{equation}

This form assumes that $t_c^n$ is a linear function of $w^n$ such that $t_c^n = t_{c_0} + (dt_c/dw) w$
\begin{equation}
	t_c^n = 
	\begin{cases}
		k_1 w^n & w^n < w_w \\
		(f_u - k_2 w_w) + k_2 w & w_w < w < w_c \\
		0 & w > w_c
	\end{cases}
\end{equation}

in which $k_1 = f_u/w_w$ and $k_2 = -f_u/(w_c - w_w)$.
\begin{equation}
t_{c_0}^n = 
\begin{cases}
(f_u - k_2 w_w) & w_w < w < w_c \\
0 & \text{otherwise}
\end{cases}
\end{equation}

and 
\begin{equation}
\dfrac{dt_c}{dw} = 
\begin{cases}
k_1 & w^n < w_w \\
k_2 & w_w < w < w_c \\
0 & w > w_c
\end{cases}
\end{equation}

The only issue is that $t_{c_0}$ and $dt_c/dw$ are dependent on $w$ as well in the piecewise linear form. 
This isn't a smooth formulation.

Discretized fluid flow
\begin{equation}
\begin{gathered}
\underbrace{\left[\int\limits_{\Omega_c}
	\mathbf{B}_f^\top k \mathbf{B}_f ds 
	\right] \mathbf{p}^n}_{\mathbf{F}^d = \mathbf{K}_{pp} \mathbf{p}^n}
+
\dfrac{1}{\Delta t^n}
\underbrace{\left[
	\int\limits_{\Omega_c} \mathbf{N}_f^\top \hat{\mathbf{n}}_{\Gamma_c}^{\top}\llbracket \mathbf{N} \rrbracket  ds 
	\right] \mathbf{d}^n}_{\mathbf{F}^s = \mathbf{K}_{sd} \mathbf{d}^n}
- 
\dfrac{1}{\Delta t^n}
\underbrace{\left[
	\int\limits_{\Omega_c} \mathbf{N}_f^\top w^{n-1} ds 
	\right]}_{\mathbf{F}^{s0}}
	\\
+
\underbrace{\left[
	\int\limits_{\Omega_c} \mathbf{N}_f^{\top} Q_{leak}^n ds
	\right]}_{\mathbf{F}^l}
+
\underbrace{\left[
	\left(\mathbf{N}_f^{\top} Q_{inj} \right) \Big\rvert_{\Gamma_q} 
	\right]}_{\mathbf{F}^f}
= \mathbf{0} 
\end{gathered}
\end{equation}

\begin{equation}
\begin{aligned}
\mathbf{R}_d^n &= \mathbf{K}_{dd} \mathbf{d}^n 
+ \mathbf{K}_{c} \mathbf{d}^n  
- \mathbf{K}_{dp} \mathbf{p}^n 
- \mathbf{K}_{ep} \mathbf{p}^n 
+ \left(\mathbf{F}^{c0}
- \mathbf{F}^{ext} \right) = \mathbf{0}
\\
\mathbf{R}_p^n &= \mathbf{K}_{pp} \mathbf{p}^n
+ \dfrac{1}{\Delta t} \mathbf{K}_{sd} \mathbf{d}^n
+ \left(\mathbf{F}^{l} 
+ \mathbf{F}^{f} 
-\mathbf{F}^{s0}\right) = \mathbf{0}
\end{aligned}
\end{equation}

In matrix form:
\begin{equation}
	\begin{bmatrix}
		\mathbf{K}_{dd} + \mathbf{K}_{c} &
		 - \mathbf{K}_{dp} - \mathbf{K}_{ep} 
		\\
		\dfrac{1}{\Delta t} \mathbf{K}_{sd} &
		\mathbf{K}_{pp}
	\end{bmatrix}
	\begin{bmatrix}
	\mathbf{d}^n
	\\
	\mathbf{p}^n
	\end{bmatrix} 
	=
	\begin{bmatrix}
	\mathbf{F}^{ext} -\mathbf{F}^{c0}
	\\
	\mathbf{F}^{s0}
	-\mathbf{F}^{l} 
	-\mathbf{F}^{f} 
	\end{bmatrix}
\end{equation}

Simplifying terms:
\begin{equation}
\begin{bmatrix}
\mathbf{K}_{d} &
- \mathbf{K}_{p} 
\\
\mathbf{K}_{sd} &
\Delta t \mathbf{K}_{pp} 
\end{bmatrix}
\begin{bmatrix}
\mathbf{d}^n
\\
\mathbf{p}^n
\end{bmatrix} 
=
\begin{bmatrix}
\mathbf{F}^{ext,d}
\\
\mathbf{F}^{ext,p} 
\end{bmatrix}
	\label{eqn:coupled}
\end{equation}

\section{Uniqueness}

The proppant concentration is completely uncoupled in the section above, therefore we will exclude it from the proof. 
Following the proof by \textcite{gupta_coupled_2016}.

\begin{equation}
	\mathbf{K}_{coupled} = 
	\begin{bmatrix}
		\mathbf{K}_{d} &
		- \mathbf{K}_{p} 
		\\
		\mathbf{K}_{sd} &
		\Delta t \mathbf{K}_{pp} 
	\end{bmatrix}
\end{equation}

\paragraph{Lemma}

Vectors $-\mathbf{K}_{p}\mathbf{p} = -(\mathbf{K}_{dp} + \mathbf{K}_{ep})\mathbf{p}$ and  $\mathbf{p}^{\top}\mathbf{K}_{sd}$ are non-zero for any non-zero vector $\mathbf{p}$.

\paragraph{Proof}
Vector $\mathbf{K}_{dp}\mathbf{p}$ represents the load on the crack surfaces corresponding to fluid pressure. 
Physcially, this load vector is non-zero for any non-zero pressure distribution on the crack faces.

Vector $\mathbf{K}_{ep}\mathbf{p}$ represents the load on the wellbore corresponding to the wellbore fluid pressure. 
Physically, this load vector is also non-zero for any non-zero wellbore pressure. 

The vector $\mathbf{K}_{sd}$ is equivalent to the transpose of $\mathbf{K}_{dp}$. 
This means that the vector $\mathbf{p}^{\top}\mathbf{K}_{sd}=\mathbf{K}_{dp} \mathbf{p}$ is also non-zero for any non-zero pressure distribution on the crack faces. 

Note that these vectors are nonzero vectors, but may have some zero terms. 

\paragraph{Theorem}
The coupled system of equations given by Equation \ref{eqn:coupled} has a unique solution if rigid body motions of the reservoir are prevented.

\paragraph{Proof}
The first set of equations in (\ref{eqn:coupled}) gives
\begin{equation}
	\mathbf{d} = \mathbf{K}_{d}^{-1}
	\left[
	\mathbf{F}^{ext,d} 
	+ \mathbf{K}_{p} \mathbf{p}
	\right]
	\label{eqn:d}
\end{equation}

Substituting this into the second set of equations in (\ref{eqn:coupled}) leads to

\begin{equation}
\begin{gathered}
	\Delta t\mathbf{K}_{pp} \mathbf{p}^n
	+ \mathbf{K}_{sd}\left \lbrace \mathbf{K}_{d}^{-1}
	\left[
	\mathbf{F}^{ext,d}
	+ \mathbf{K}_{p} \mathbf{p}
	\right]\right \rbrace
	- \mathbf{F}^{ext,p} = \mathbf{0}
	\\
	\Delta t \mathbf{K}_{pp} \mathbf{p}^n
	+ \mathbf{K}_{sd} \mathbf{K}_{d}^{-1}
	\mathbf{F}^{ext,d}
	+ \mathbf{K}_{sd} \mathbf{K}_{d}^{-1} \mathbf{K}_{p} \mathbf{p}
	- \mathbf{F}^{ext,p} = \mathbf{0}
	\\
	\underbrace{\left(\Delta t \mathbf{K}_{pp} + \mathbf{K}_{sd} \mathbf{K}_{d}^{-1} \mathbf{K}_{p} \right)}_{\mathbf{S}_k}
	\mathbf{p}^n
	= \mathbf{F}^{ext,p}
	- \mathbf{K}_{sd} \mathbf{K}_{d}^{-1}
	\mathbf{F}^{ext,d} 
\end{gathered}
\label{eqn:p}
\end{equation}

in which $\mathbf{S}_k$ is the Schur complement of $\mathbf{K}_{d}$ in $\mathbf{K}_{coupled}$. 

Sub-matrix $\mathbf{K}_{d}$ is positive definite if rigid body motions of the reservoir are prevented.
Thus, this proves nonsingularity of $\mathbf{K}_{d}$ and Equation \ref{eqn:d} gives a unique solution for $\mathbf{d}$. 

Similarly, Equation \ref{eqn:p} gives a unique solution for $\mathbf{p}$ if the Schur complement, $\mathbf{S}_k$ is positive definite. 
To prove that, pre- and post- multiply the Schur complement by an arbitrary non-zero vector, $\mathbf{p}$, 

\begin{equation}
\begin{aligned}
	\mathbf{p}^{\top} \mathbf{S}_k \mathbf{p} &= 
\mathbf{p}^{\top} \left(\Delta t\mathbf{K}_{pp} + \mathbf{K}_{sd} \mathbf{K}_{d}^{-1} \mathbf{K}_{p} \right) \mathbf{p}
\\
&= 
\mathbf{p}^{\top} \Delta t\mathbf{K}_{pp} \mathbf{p}
+
\mathbf{p}^{\top} \mathbf{K}_{sd} \mathbf{K}_{d}^{-1} \mathbf{K}_{p} \mathbf{p}
\\
&= 
\mathbf{p}^{\top} \Delta t\mathbf{K}_{pp} \mathbf{p}
+
\mathbf{p}^{\top} \mathbf{K}_{dp}^{\top} \mathbf{K}_{d}^{-1} \left(\mathbf{K}_{dp} + \mathbf{K}_{ep}\right) \mathbf{p}
\\
&=
\mathbf{p}^{\top} \Delta t\mathbf{K}_{pp} \mathbf{p}
+
\left(\mathbf{p}^{\top} \mathbf{K}_{dp}^{\top}\right) \mathbf{K}_{d}^{-1} \left(\mathbf{K}_{dp} \mathbf{p}\right)
+
\mathbf{p}^{\top} \left( \mathbf{K}_{dp}^{\top} \mathbf{K}_{d}^{-1} \mathbf{K}_{ep} \right) \mathbf{p}
\end{aligned}
\end{equation}

The Schur complement is positive definite if $\mathbf{p}^{\top} \mathbf{S}_k \mathbf{p}$ is greater than zero.
The term $\mathbf{p}^{\top} \Delta t\mathbf{K}_{pp} \mathbf{p}$ is greater than or equal to zero for all non-zero $\Delta t$ and non-zero vector $\mathbf{p}$ because $\mathbf{K}_{pp}$ is positive semi-definite.

Because $\mathbf{K}_{d}^{-1}$ is positive definite and $\mathbf{K}_{dp} \mathbf{p}$ is non-zero, the second term $\mathbf{p}^{\top} \mathbf{K}_{dp} \mathbf{K}_{d}^{-1} \mathbf{K}_{dp} \mathbf{p}$ is greater than zero for any non-zero vector $\mathbf{p}$.

The last term is asymmetric, but still non-zero as long as $\mathbf{p}^{\top} \mathbf{K}_{dp}^{\top}$ and $\mathbf{K}_{ep} \mathbf{p}$ are non-zero. The first vector has already been shown to be non-zero, and the second vector is non-zero as long as the inlet pressure, $p_w$ is non-zero.
Thus, the solution of the coupled system of equations is unique if rigid body motions of the reservoir are prevented.

\section{Linearization}
Linearization
\begin{equation}
\mathbf{R}^n_m + \mathbf{J}^n_m \Delta \boldsymbol{\psi}^n_m = \mathbf{0}
\end{equation}

Jacobian 
\begin{equation}
\mathbf{J}^n_m = 
\begin{bmatrix}
\dfrac{\partial \mathbf{R}_d^n}{\partial \mathbf{d}^n}
& 
\dfrac{\partial \mathbf{R}_d^n}{\partial \mathbf{p}^n}
\\[1.2em]
\dfrac{\partial \mathbf{R}_p^n}{\partial \mathbf{d}^n}
& 
\dfrac{\partial \mathbf{R}_p^n}{\partial \mathbf{p}^n}
\end{bmatrix}_m
= \begin{bmatrix}
\mathbf{K}_{dd} + \mathbf{K}_{cd}
& 
-\mathbf{K}_{dp} - \mathbf{K}_{ep}
\\[1.2em]
\Delta t\mathbf{K}_{pd} + \mathbf{K}_{sd}
& 
\Delta t\mathbf{K}_{pp}
\end{bmatrix}^n_m
\end{equation}	

The terms in the matrix are
\begin{align}
	\mathbf{K}_{dd} &= \dfrac{\partial \mathbf{F}^{int}}{\partial \mathbf{d}^n}  
	= \int\limits_{\Omega} \mathbf{B}^{\top} \mathbf{D} \mathbf{B} d\Omega
	\\[1.2em]
	\mathbf{K}_{cd} &= \dfrac{\partial \mathbf{F}^{c}}{\partial \mathbf{d}^n} 	
	= \dfrac{\partial \mathbf{F}^{c}}{\partial t_c} \dfrac{\partial t_c}{\partial w} 
	\dfrac{\partial w}{\partial \mathbf{d}^n} \nonumber
	\\
	&=\left[\int\limits_{\Gamma_{c}} \llbracket\mathbf{N}\rrbracket ^{\top} \hat{\mathbf{n}}_{\Gamma_c} 
	\dfrac{\partial t_c}{\partial w}  \hat{\mathbf{n}}_{\Gamma_c}^\top \llbracket \mathbf{N} \rrbracket  d\Gamma \right]
	\\[1.2em]
	\mathbf{K}_{dp} &= \dfrac{\partial \mathbf{F}^{p}}{\partial \mathbf{p}^n}
	= \int\limits_{\Gamma_{c}} \llbracket \mathbf{N} \rrbracket^{\top} \hat{\mathbf{n}}_{\Gamma_c} \mathbf{N}_f d\Gamma 
	\\[1.2em]
	\mathbf{K}_{ep} &= \dfrac{\partial \mathbf{F}^{ext}}{\partial \mathbf{p}^n}
	= \int\limits_{\Gamma_w} \mathbf{N}^{\top} \hat{\mathbf{n}}_{w} d\Gamma \:
	\bar{\mathbf{I}}_1^{\top}
	\\[1.2em]
	\mathbf{K}_{pd} &= \dfrac{\partial \mathbf{F}^d}{\partial \mathbf{d}^n}   
	= \dfrac{\partial \mathbf{F}^d}{\partial k}
	\dfrac{\partial k}{\partial w} 
	\dfrac{\partial w}{\partial \mathbf{d}^n} \nonumber
	\\
	&=  \left[\int\limits_{\Omega_f}
	\mathbf{B}_f^{\top}
	\dfrac{\partial k}{\partial w}  \mathbf{B}_f \mathbf{p}^n \hat{\mathbf{n}}_{\Gamma_c}^\top \llbracket \mathbf{N} \rrbracket ds \right] 		
	\\[1.2em]
	\mathbf{K}_{sd} &= 
	\dfrac{\partial \mathbf{F}^s}{\partial \mathbf{d}^n} 
	= \dfrac{\partial \mathbf{F}^s}{\partial w^n} \dfrac{\partial w^n}{\partial \mathbf{d}^n} \nonumber
	\\
	&= \int\limits_{\Omega_f} \mathbf{N}_f^{\top} \hat{\mathbf{n}}_{\Gamma_c}^{\top} \llbracket \mathbf{N} \rrbracket ds
	\\[1.2em]
	\mathbf{K}_{pp} &= \dfrac{\partial \mathbf{F}^{d}}{\partial \mathbf{p}^n}
	= \dfrac{\partial \mathbf{F}^{d}}{\partial q}
	\dfrac{\partial q}{\partial \mathbf{p}^n} \nonumber
	\\
	&= \left[\int\limits_{\Omega_f} \mathbf{B}_f^{\top} k\mathbf{B}_f ds \right] 
\end{align}


The differential terms in the integrals are specified as,
\begin{align}
	\dfrac{\partial t_c}{\partial w}&= \begin{cases}
		\dfrac{f_u}{w_w}, & w < w_w
		\\[1.2em]
		\dfrac{-f_u}{w_c-w_w}, & w_w < w < w_c
		\\[1.2em]
		0, & w > w_c
	\end{cases}
	\\[1.2em]
	\dfrac{\partial k}{\partial w} &= \dfrac{w^2}{4 \mu}
\end{align}

Newton-Raphson iterative method

\begin{equation}
\boldsymbol{\psi}^{n}_{m} = \boldsymbol{\psi}^{n}_{m-1} + \Delta \boldsymbol{\psi}^{n}_{m-1}
\label{eqn:solve}
\end{equation}

Convergence criteria

\begin{equation}
\lVert \mathbf{R}^n_m \rVert \leq \epsilon_{tol,a} 
\quad \text{or} \quad 
\lVert\mathbf{R}^n_m \rVert \leq \epsilon_{tol,b} \lVert \mathbf{R}^n_1 \rVert
\label{eqn:conv-criteria}
\end{equation}

% \section{Uniqueness of Linearized Form}

% \begin{equation}
% 	\begin{bmatrix}
% 	\mathbf{R}_d \\[1.2em]
% 	\mathbf{R}_p 
% 	\end{bmatrix}
% 	+ 
% 	\begin{bmatrix}
% 	\mathbf{K}_{dd} + \mathbf{K}_{cd}
% 	& 
% 	-\mathbf{K}_{dp} - \mathbf{K}_{ep}
% 	\\[1.2em]
% 	\Delta t\mathbf{K}_{pd} +\mathbf{K}_{sd}
% 	& 
% 	\Delta t\mathbf{K}_{pp}
% 	\end{bmatrix}
% 	\begin{bmatrix}
% 	\Delta\mathbf{d} \\[1.2em]
% 	\Delta\mathbf{p}
% 	\end{bmatrix}
% 	=
% 	\mathbf{0}
% \end{equation}

% Simplify as
% \begin{equation}
% \begin{bmatrix}
% \mathbf{K}_{d}
% & 
% -\mathbf{K}_{p}
% \\[1.2em]
% \mathbf{K}_{p2}
% & 
% \mathbf{K}_{pp}
% \end{bmatrix}
% \begin{bmatrix}
% \Delta\mathbf{d} \\[1.2em]
% \Delta\mathbf{p} 
% \end{bmatrix}
% =
% \begin{bmatrix}
% -\mathbf{R}_d \\[1.2em]
% -\mathbf{R}_p 
% \end{bmatrix}
% \end{equation}

% From the first equation, 

% \begin{gather}
% 	\mathbf{K}_{d}\Delta\mathbf{d} 
% 	- \mathbf{K}_{p} \Delta\mathbf{p} 
% 	= -\mathbf{R}_d
% 	\\
% 	\Delta\mathbf{d} = \mathbf{K}_{d}^{-1} \left(-\mathbf{R}_d + \mathbf{K}_{p} \Delta\mathbf{p} \right)
% 	\\
% 	\Delta\mathbf{d} = -\mathbf{K}_{d}^{-1} \mathbf{R}_d + \mathbf{K}_{d}^{-1}\mathbf{K}_{p} \Delta\mathbf{p}
% \end{gather}

% This yields a unique solution so long as $\mathbf{K}_{d}=\mathbf{K}_{dd} + \mathbf{K}_{cd}$ is invertible, proven by positive definiteness.
% This is true as long as rigid body motions are restrained. 

% Substituting into the second equation yields
% \begin{gather}
% 	\mathbf{K}_{p2}\Delta\mathbf{d} 
% 	+ \mathbf{K}_{pp}\Delta \mathbf{p} 
% 	= -\mathbf{R}_p
% 	\\
% 	\mathbf{K}_{pd2}\left(-\mathbf{K}_{d}^{-1} \mathbf{R}_d + \mathbf{K}_{d}^{-1}\mathbf{K}_{p} \Delta \mathbf{p}\right) 
% 	+ \mathbf{K}_{pp} \Delta \mathbf{p} 
% 	= -\mathbf{R}_p
% 	\\
% 	-\mathbf{K}_{pd2}\mathbf{K}_{d}^{-1} \mathbf{R}_d 
% 	+ \mathbf{K}_{pd2}\mathbf{K}_{d}^{-1}\mathbf{K}_{p}\Delta\mathbf{p} 
% 	+ \mathbf{K}_{pp} \Delta \mathbf{p} 
% 	=-\mathbf{R}_p
% 	\\ 
% 	\underbrace{\left(\mathbf{K}_{pd2}\mathbf{K}_{d}^{-1}\mathbf{K}_{p}+ \mathbf{K}_{pp}  \right)}_{\mathbf{S}_k}\Delta \mathbf{p} 
% 	= \mathbf{K}_{pd2}\mathbf{K}_{d}^{-1} \mathbf{R}_d -\mathbf{R}_p 
% \end{gather}

% This yields a unique solution so long as $\mathbf{S}_k = \mathbf{K}_{pd2}\mathbf{K}_{d}^{-1}\mathbf{K}_{p}+ \mathbf{K}_{pp} $ is invertible. Same as above, we prove this by showing that $\mathbf{p}^{\top}\mathbf{S}_k\mathbf{p}$ is greater than zero.


\end{document}
